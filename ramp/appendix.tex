
\newcommand{\transform}{\textsc{Transform}\xspace}

\section{RSIW Proof}

\label{sec:rsiwproof}


% total order valid because G1c is prohibited

To begin, we first show that there exists a well-defined total
ordering of write-only transactions in a history that is valid under
RA isolation. This will be useful in ordering write transactions in
our one-copy equivalent execution.

\begin{lemma}[Well-defined Total Order on Writes]
\label{lemma:to}
  Given a history $H$ containing read-only and write-only transactions
  that is valid under RA isolation, $DSG(H)$ does not contain any
  directed cycles consisting entirely of write-dependency edges.
\end{lemma}
\begin{proof}
  $H$ is valid under RA isolation and therefore does not exhibit
  phenomenon \textit{G1c}. Thus, $H$ does not contain any directed
  cycles consisting entirely of dependency edges. Therefore, $H$ does
  not contain any directed cycles consisting entirely of
  write-dependency edges.
\end{proof}

We will also need to place read-only transactions in our history. To
do so, we show that, under RA isolation and the RSIW property (i.e.,
the preconditions of Theorem~\ref{thm:rsw}), each read-only
transaction will only read from one write-only transaction.

\begin{lemma}[Single Read Dependency]
\label{lemma:onedep}
Given a history $H$ containing read-only and write-only transactions
that obeys the RSIW property and is valid under RA isolation, each
node in $DSG(H)$ contains at most one direct read-dependency edge.
\end{lemma}
\begin{proof}
  Consider a history $H$ containing read-only and write-only
  transactions that has RSIW and is valid under RA
  isolation. Write-only transactions have no reads, so they have no
  read-dependency edges in $DSG(H)$. However, suppose $DSG(H)$
  contained a node corresponding to a read-only transaction containing
  more than one direct read-dependency edge; call this read-only
  transaction $T_r$. For two read-dependency edges to exist, $T_r$
  must have read versions produced by at least two different
  write-only transactions; pick any two and call them $T_i$ and $T_j$,
  corresponding to read versions $x_a$ and $y_d$.

  If $x$ and $y$ are the same item, then $a < d$ or $d < a$. In either
  case, $T_r$ exhibits the fractured reads phenomenon and $H$ is not
  valid under RA isolation, a contradiction.

  Therefore, $x$ and $y$ must be distinct items. Because $H$ obeys the
  RSIW property, $T_r$ must also obey the RSIW property. By the
  definition of the RSIW property, $T_r$ must have only read items
  written to by $T_i$ and items also written to by $T_j$; this implies
  that $T_i$ and $T_j$ each wrote to items $x$ and $y$. We can label
  $T_i$'s write to $y$ as $y_b$ and $T_j$'s write to $x$ as $x_c$. Per
  Lemma~\ref{lemma:to}, $T_i$'s writes to $x$ and $y$ must either both
  come before or both follow $T_j$'s corresponding writes to $x$ and
  $y$ in the version order for each of $x$ and $y$; that is, either
  both $a < b$ and $c < d$ or both $b < a$ and $d < c$.

  If $a < b$ and $c < d$, then $T_r$ exhibits the fractured reads
  phenomenon: $T_r$ read $x_a$ and $y_d$ but $T_j$, which wrote $y_d$
  also wrote $x_b$, and $a < b$. If $b < a$ and $d < c$, then $T_r$
  again exhibits the fractured reads phenomenon: $T_r$ read $x_a$ and
  $y_d$ but $T_i$, which wrote $x_a$, also wrote $y_c$, and $d <
  c$. In either case, $H$ is not valid under RA isolation, a
  contradiction.

  Therefore, each node in $DSG(H)$ must not contain more than one
  read-dependency edge.
\end{proof}

We now use this ordering on reads and writes to construct a total
ordering on transactions in a history:

\begin{procedure}[Transform]
  Given a history $H$ containing read-only and write-only transactions
  that has RSIW and is valid under RA isolation,
  construct a total ordering $O$ of the transactions in $H$ as
  follows:
\begin{enumerate}
\item Perform a topological sorting in $O$ of committed write-only
  transactions in $H$ ordered by the write-dependency edges in
  $DSG(H)$. That is, for each pair of write-only transactions ($T_1$,
  $T_2$) in $H$ that performed at least one write to the same item,
  place the transaction that wrote the higher-versioned item later in
  $O$. Lemma~\ref{lemma:to} ensures such a total ordering exists.
\item For each committed read-only transaction $T_r$ in $H$, place
  $T_r$ in $O$ after the write-only transaction $T_w$ whose writes
  $T_r$ read (i.e., after the write-only transaction that $T_r$
  directly read-depends on) but before the next write-only transaction
  $T_{w'}$ in $O$ (or, if no such transaction exists, at the end of
  $O$). By Lemma~\ref{lemma:onedep}, each committed read-only
  transaction read-depends on only one write-only transaction, so this
  ordering is similarly well defined.
\end{enumerate}

Return $O$.

\end{procedure}

As an example, consider the following history:
\begin{eqnarray}
\label{hist:transform-example}
T_1 & w(x_1); w(y_1)\\
T_2 & w(x_2); w(y_2)\nonumber\\
T_3 & r(x_1); r(y_1)\nonumber\\
T_4 & r(y_2);\nonumber
\end{eqnarray}
History~\ref{hist:transform-example} obeys the RSIW property and is
also valid under RA isolation. Applying procedure~\transform to
History~\ref{hist:transform-example}, in the first step, we first
order our write-only transactions $T_1$ and $T_2$. Both $T_1$ and
$T_2$ write to $x$ and $y$, but $T_2$'s writes have a later version
number than $T_1$'s, so, according to Step 1 of~\transform, we have
$O=T_1; T_2$. Now, in Step 2 of~\transform, we consider the read-only
transactions $T_3$ and $T_4$. We place $T_3$ after the transaction
that it read from ($T_1$) and before the next write transaction in the
sequence ($T_2$), yielding $O=T_1; T_3; T_2$. We place $T_4$ after the
transaction that it read from ($T_2$) and, because there is no later
write-only transaction following $T_2$ in $O$, place $T_4$ at the end
of $O$, yielding $O=T_1; T_3; T_2; T_4$. In this case, we observe
that, as Theorem~\ref{thm:rsw} suggests, it is possible to~\transform
an RSIW and RA history into a one-copy serial equivalent and that $O$ is in fact
a one-copy serializable execution.

Now we can prove Theorem~\ref{thm:rsw}. We demonstrate that executing
the transactions of $H$ in the order resulting from
applying~\transform to $H$ on a single-copy database yields an
equivalent history (i.e., read values and final database state) as
$H$. Because $O$ is a total order, $H$ must be one-copy serializable.
\begin{proof}[Theorem~\ref{thm:rsw}]
  Consider history $H$ containing read-only and write-only
  transactions that has RSIW and is valid under RA
  isolation. We begin by applying~\transform to $H$ to produce an
  ordering $O$. 

  We create a new history $H_o$ by considering an empty one-copy
  database and examining each transaction $T_h$ in $O$ in serial order
  as follows: if $T_h$ is a write-only transaction, execute a new
  transaction $T_{ow}$ that writes each version produced by $T_h$ to
  the one-copy database and commits. If $T_h$ is a read-only
  transaction, execute a new transaction $T_{or}$ that reads from the
  sam items as $T_h$ and commits. $H_o$ is the result of a serial
  execution of transactions over a single logical copy of the
  database, $H_o$ is one-copy serializable.

  We now show that $H$ and $H_o$ are equivalent. First, all committed
  transactions and operations in $H$ also appear in $H_o$
  because~\transform operates on all transactions in $H$ and all
  transactions and their operations (with single-copy operations
  substituted for multi-version operations) appear in the total order
  $O$ used to produce $H_o$. Second, $DSG(H)$ and $DSG(H_o)$ have the
  same direct read dependencies. This is a straightforward consequence
  of Step 2 of~\transform: in $O$, each read-only transaction $T_r$
  appears immediately following the write-only transaction $T_w$ upon
  which $T_r$ read-depends. When the corresponding read transaction is
  executed against the single-copy database in $H_o$, the serially
  preceding write-only transaction will produce the same values that
  the read transaction read in $H$. Therefore, $H$ and $H_o$ are equivalent.

  Because $H_o$ is one-copy serializable and $H_o$ is equivalent to $H$, $H$
  must also be one-copy serializable.
\end{proof}

We have opted for the above proof technique because we believe
the~\transform procedure provides clarity into \textit{how} executions
satisfying both RA isolation and the RSIW property relate to their
serializable counterparts. An alternative and elegant proof approach
leverages the work on multi-version serializability
theory~\cite{bernstein-book}, which we sketch here. Given a
history $H$ that exhibits RA isolation and has RSIW, we show
that $MVSG(H)$ is acyclic. By an argument resembling
Lemma~\ref{lemma:onedep}, the in-degree for read-only transactions in
$SG(H)$ (i.e., Adya's direct read dependencies) is one. By an argument
resembling Lemma~\ref{lemma:to}, the edges between write-only
transactions in $MVSG(H)$ produced by the first condition of the
$MVSG$ construction ($x_i \ll x_j$ in the definition of the
$MVSG$~\cite[p. 152]{bernstein-book}; i.e., Adya's write dependencies)
are acyclic. Therefore, any cycles in the $MVSG$ must include at least
one of the second kind of edges in the $MVSG(H)$ ($x_j \ll x_i$; i.e.,
Adya's direct anti-dependencies). But, for such a cycle to exist, a
read-only transaction $T_r$ must anti-depend on a write-only
transaction $T_{wi}$ that in turn write-depends on another write-only
transaction $T_{wj}$ upon which $T_r$ read-depends. Under the RSIW
property, $T_{wi}$ and $T_{wj}$ will have written to at least one of
the same items, and the presence of a write-dependency cycle will
indicate a fractured reads anomaly in $T_r$.

\section{RAMP Correctness and Independence}

\label{sec:rampproofs}


\minihead{RAMP-F Correctness} To prove \rapl provides RA isolation, we
show that the two-round read protocol returns a transactionally atomic
set of versions. To do so, we formalize criteria for atomic (read)
sets of versions in the form of \textit{companion sets}. We will call
the set of versions produced by a transaction \textit{sibling
  versions} and say that $x$ is a sibling item to a write $y_j$ if
there exists a version $x_k$ that was written in the same transaction
as $y_j$.

Given two versions $x_i$ and $y_j$, we say that $x_i$ is a
\textit{companion version} to $y_j$ if $x_i$ is a transactional
sibling of $y_j$ or $x$ is a sibling item of $y_j$ and $i > j$. We say
that a set of versions $V$ is a \textit{companion set} if, for every
pair $(x_i, y_j)$ of versions in $V$ where $x$ is a sibling item of
$y_j$, $x_i$ is a companion to $y_j$. In
Figure~\ref{fig:rapl-execution}, the versions returned by $T_2$'s
first round of reads ($\{x_1, y_\bot\}$) do not comprise a companion
set because $y_\bot$ has a lower timestamp than $x_1$'s sibling
version of $y$ (that is, $x_1$ has sibling version $y_1$ and but $\bot
< 1$ so $y_\bot$ has too low of a timestamp). Subsets of companion
sets are also companion sets and companion sets also have a useful
property for RA isolation:
\begin{claim}[Companion sets are atomic]
\label{claim:companion-atomic}
In the absence of G1c phenomena, companion sets do not contain fractured reads.
\end{claim}

Claim~\ref{claim:companion-atomic} follows from the definitions of
companion sets and fractured reads.

\begin{proof}
If $V$ is a companion set, then
every version $x_i \in V$ is a companion to every other version
$y_j \in V$ where $v_j$ contains $x$ in its sibling items. If $V$
contained fractured reads, $V$ would contain two versions $x_i,y_j$
such that the transaction that wrote $y_j$ also wrote a version $x_k$,
$i<k$. However, in this case, $x_i$ would not be a companion to
$y_j$, a contradiction. Therefore, $V$ cannot contain fractured reads.
\end{proof}

To provide RA, \rapl clients assemble a companion set for the
requested items (in $v_{latest}$), which we prove below:

\begin{claim}\rapl provides Read Atomic isolation.\end{claim}

\begin{proof}
\label{proof:rapl-ra}
Each write in \rapl contains information regarding its siblings, which
can be identified by item and timestamp. Given a set of \rapl
versions, recording the highest timestamped version of each item (as
recorded either in the version itself or via sibling metadata) yields
a companion set of item-timestamp pairs: if a client reads two
versions $x_i$ and $y_j$ such that $x$ is in $y_j$'s sibling items but
$i < j$, then $v_{latest}[x]$ will contain $j$ and not
$i$. Accordingly, given the versions returned by the first round of
\rapl reads, clients calculate a companion set containing versions of
the requested items. Given this companion set, clients check the
first-round versions against this set by timestamp and issue a second
round of reads to fetch any companions that were not returned in the
first round. The resulting set of versions will be a subset of the
computed companion set and will therefore also be a companion
set. This ensures that the returned results do not contain fractured
reads. \rapl first-round reads access $lastCommit$, so each
transaction corresponding to a first-round version is committed, and,
therefore, any siblings requested in the (optional) second round of
reads are also committed. Accordingly, \rapl never reads aborted or
non-final (intermediate) writes. Moreover, \rapl timestamps are
assigned on a per-transaction basis, preventing write-dependency
cycles and therefore \textit{G1c}. This establishes that \rapl
provides RA\@.
\end{proof}

\minihead{RAMP-F Scalability and Independence} \rapl also provides the
independence guarantees from Section~\ref{sec:sysmodel}. The
following invariant over $lastCommit$ is core to \rapl \textsc{get}
request completion:
\begin{invariant}[Companions present]
\label{inv:suitable-present}

If a version $x_i$ is referenced by $lastCommit$ (that is,
$lastCommit[x] = i$), then each of $x_i$'s sibling versions are
present in $versions$ on their respective partitions.

\end{invariant}
Invariant~\ref{inv:suitable-present} is maintained by \rapl's
two-phase write protocol. $lastCommit$ is only updated once a
transaction's writes have been placed into $versions$ by a first round
of \textsc{prepare} messages. Siblings will be present in $versions$
(but not necessarily $lastCommit$).

\begin{claim}\rapl is coordination-free.\end{claim}

Recall from Section~\ref{sec:sysmodel} that coordination-free execution ensures that one client's
transactions cannot cause another client's to block and that, if a
client can contact the partition responsible for each item in its
transaction, the transaction will eventually commit (or abort of its
own volition).

\begin{proof}
Clients in \rapl do not communicate or coordinate with one another and
only contact servers. Accordingly, to show that \rapl provides
coordination-free execution, it suffices to show that server-side
operations always terminate. \textsc{prepare} and \textsc{commit}
methods only access data stored on the local partition and do not
block due to external coordination or other method invocations;
therefore, they complete. \textsc{get} requests issued in the first
round of reads have $ts_{req} = \bot$ and therefore will return the
version corresponding to $lastCommit[k]$, which was placed into
$versions$ in a previously completed \textsc{prepare}
round. \textsc{get} requests issued in the second round of client
reads have $ts_{req}$ set to the client's calculated
$v_{latest}[k]$. $v_{latest}[k]$ is a sibling of a version returned
from $lastCommit$ in the first round, so, due to
Invariant~\ref{inv:suitable-present}, the requested version will be
present in $versions$. Therefore, \textsc{get} invocations are
guaranteed access to their requested version and can return without
waiting. The success of \rapl operations does not depend on the success
or failure of other clients' \rapl operations.
\end{proof}

\begin{claim}\rapl provides partition independence.\end{claim}
\begin{proof}
\rapl transactions do not access partitions that are unrelated to each transaction's
specified data items and servers do not contact other servers in
order to provide a safe response for operations.
\end{proof}

\minihead{RAMP-S Correctness} \raps writes and first-round reads proceed
identically to \rapl writes, but the metadata written and returned is
different. Therefore, the proof is similar to \rapl, with a slight
modification for the second round of reads.

\begin{claim}\raps provides Read Atomic isolation.\end{claim}
\begin{proof}
  To show that \raps provides RA, it suffices to show that \raps
  second-round reads ($resp$) are a companion set. Given two versions
  $x_i, y_j \in resp$ such that $x \neq y$, if $x$ is a sibling item
  of $y_j$, then $x_i$ must be a companion to $y_j$. If $x_i$ were not
  a companion to $y_j$, then it would imply that $x$ is not a sibling
  item of $y_j$ (so we are done) or that $j > i$. If $j > i$, then,
  due to Invariant~\ref{inv:suitable-present} (which also holds for
  \raps writes due to identical write protocols), $y_j$'s sibling is
  present in $versions$ on the partition for $x$ and would have been
  returned by the server (line~\ref{raps-get-server-withts-end}), a
  contradiction. Each second-round \textsc{get} request returns only
  one version, so the final set of reads does not exhibit fractured reads.
\end{proof}

\minihead{RAMP-S Scalability and Independence} \raps ensures
coordination-free execution and partition independence. The proofs
closely resemble those of \rapl: Invariant~\ref{inv:suitable-present}
ensures that incomplete commits do not stall readers, and all
server-side operations are guaranteed to complete.

\minihead{RAMP-H Correctness} The probabilistic behavior of the \rapb Bloom
filter admits false positives. However, given unique transaction
timestamps (Section~\ref{sec:additional}), requesting false siblings
by timestamp and item does not affect correctness:

\begin{claim}\rapb provides Read Atomic isolation.\end{claim}
\begin{proof}
To show that \rapb provides Read Atomic isolation, it suffices to show
that any versions requested by \rapb second-round reads that would not
have been requested by \rapl second-round reads (call this set
$v_{false}$) do not compromise the validity of \rapb's returned
companion set. Any versions in $v_{false}$ do not exist: timestamps
are unique, so, for each version $x_i$, there are no versions $x_j$ of
non-sibling items with the same timestamp as $x_i$ (i.e., where
$i=j$). Therefore, requesting versions in $v_{false}$ do not change
the set of results collected in the second round.
\end{proof}


\minihead{RAMP-H Scalability and Independence} \rapb provides
coordination-free execution and partition independence. We omit full
proofs, which closely resemble those of \rapl. The only significant
difference from \rapl is that second-round \textsc{get} requests may
return $\bot$, but, as we showed above, these empty responses
correspond to false positives in the Bloom filter and therefore do not
affect correctness.

