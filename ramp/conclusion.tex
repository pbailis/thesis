
\section{Discussion}

Given our experiences designing and evaluating the RAMP transaction
protocols, we believe there are a number of interesting extensions
that merit further examination.

First, RAMP metadata effectively encodes a limited form of transaction
\textit{intent}: readers and servers are only able to repair fractured
reads because the metadata encodes the remainder of the work required
to complete the transaction. We believe it would be interesting to
generalize this concept to arbitrary program logic: for example, in a
model such as lazy transactions~\cite{lazytxn} or eventual
serializability~\cite{eventual-serial}, with transactions expressed as
stored procedures, multiple, otherwise conflicting/coordinating
clients could instead cooperate in order to complete one anothers'
transactions in the event of a failure---without resorting to the use
of a centralized master (e.g., for pre-scheduling or validating transaction
execution). This programming model is largely incompatible with
traditional interactive transaction execution but is nevertheless
exciting to consider as an extension of these protocols.

Second, and more concretely, we see several opportunities to extend
RAMP to more specialized use cases. The RAMP protocol family is
currently not well-suited to large scans and, as we have discussed,
does not enforce transitive dependencies across transactions. We view
restricting the concurrency of writers (but not readers) to be a
useful step forward in this area, with predictable impact on writer
performance. This strikes a middle ground between traditional MVCC and
the current RAMP protocols.

Finally, as we noted in Section~\ref{sec:ra-serializable}, efficient
transaction processing often focuses on weakening semantics (e.g.,
weak isolation) or changing the programming model (e.g., stored
procedures as above). As our investigation of the RSW property
demonstrates, there may exist compelling combinations of the two that
yield more intuitive, high-performance, or scalable results than
examining semantics or programming models in isolation. Addressing
this question is especially salient for the many users of weak
isolation models in practice today~\cite{hat-vldb}, as it can help
understand when applications require stronger semantics and when, in
fact, weak isolation is not simply fast but is also ``correct.''

\section{Summary}
\label{sec:conclusion}

This chapter described how to achieve atomically visible multi-partition
transactions without incurring the performance and availability
penalties of traditional algorithms. We first identified a new
isolation level---Read Atomic isolation---that provides atomic
visibility and matches the requirements of a large class of real-world
applications. We subsequently achieved RA isolation via scalable,
contention-agnostic RAMP transactions. In contrast with techniques
that use inconsistent but fast updates, RAMP transactions provide
correct semantics for applications requiring secondary indexing,
foreign key constraints, and materialized view maintenance while
maintaining scalability and performance. By leveraging
multi-versioning with a variable but small (and, in two of three
algorithms, constant) amount of metadata per write, RAMP transactions
allow clients to detect and assemble atomic sets of versions in one to
two rounds of communication with servers (depending on the RAMP
implementation). The choice of coordination-free and partition
independent algorithms allowed us to achieve near-baseline performance
across a variety of workload configurations and scale linearly to 100
servers. While RAMP transactions are not appropriate for all
applications, the many applications for which they are appropriate will
benefit significantly.
